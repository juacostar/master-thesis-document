\chapter{Related Work}
\label{cha:relatedwork}

\section{Research Methodology}
For architectural erosion symptoms, causes, and consequences detection, during the research, it was necessary to realize a series of steps for identification and possible recovery processes. Due to the orientation of the actual architectural erosion solvers for solving them mostly for Backend and Frontend projects, it is necessary to identify architectural erosion symptoms in Android apps. After that, with human judges, these symptoms have to be confirmed, and, finally, those symptoms will generate architectural rules for detecting them and suggest different recovering ways during the coding stage of a system.

\subsection{Architectural Erosion: An Initial Overview}
As the first approach is to define, and explain the concept of architectural erosion in software engineering and to set the relationship between this concept and the mobile development ecosystem. In the first approach, we look for advances in static analysis solution approach in architecture quality gates in Server-Side and Frontend applications \emph{reference paper uniandes}. In this case, we locate the cited related work in this research to search for the initial motivation for solving architectural erosion issues and find quality gates in terms of performance in the Android ecosystem.

The first overview we extracted about the related in \citet{slr-base}. In this paper, we find a Systematic Literature Review (SLR) that defines the definition, reasons, symptoms, consequences, and solution approaches to architectural erosion. In the resume, this approach finds 73 relevant papers about 8 research questions related to the last mentioned features. From this approach, we can execute the same query in different research papers databases for more actual research, due to the publication year of this research (2021), and the time period criteria (between 2006 and 2009) for finding recent research and advances during the last three years. The paper searches relevant papers in 7 research databases and makes a better-performed query, due to the relation between the "architectural erosion" concept in civil engineering and a phenomenon presented in buildings. We present the first used search queries and its databases:

\begin{table}[H]
    \centering
    \begin{tabular}{|c|}
        \hline
        Query\\
         \hline
             ("software" OR "software system" OR "software engineering") \\
             AND ("architecture" OR "architectural structure" OR "structural") \\
             AND ("erosion" OR "decay" OR "degradation" OR "deterioration"\\
             OR "degeneration") \\
         \hline
    \end{tabular}
    \caption{Executed Query for related work search}
    \label{tab:my_label}
\end{table}

Since those first results, we can extract relevant papers that include new developments and approaches for fixing architectural erosion issues, metrics, tools, and more related work. With this, we identify in each paper three main stages for solving those issues. In the detection stage, we use different identification alternatives through developer messages in versioning systems like GitHub. The Detection Stage, where we use Model Driven Development (MDD) and code patterns detection for detecting the identified architectural violations rules. Finally, based on the solution approach (Design approach, quality approach, etc.), we can suggest a proposal for solving the detected architectural violations.
In the first search iteration, we added another filter related to the publication year. We selected the papers whose publication year is between 2021 and 2024, that papers include in their bibliography the first SLR that use as reference research. Despite different troubles with the query, when different databases show papers related to civil engineering or architecture (this is because the architectural erosion definition in that context is another research topic about building degradation). In this first research iteration, we found researches that synthesize different solution approaches and give an overview from different perspectives, since the process of AER issues detection, metrics that could indicate a possible architectural violation in a software project like coupling metrics and the relationship between classes.
We observed that the use of NLP techniques could improve the performance of AER issues detection based on commit messages analysis. With the use of pre-trained Word Embedding models trained in a specific context, different AER issues are detected, and a list of potential keywords could be generated as a type of alert of an architectural violation based on the architectural model base.
In the symptoms and causes approach, an analysis of different base applications made in different programming languages was made. Different alternatives and methodologies were evaluated to determine the performance in architectural erosion issues detection and different solution approaches were proposed.

\section{Architectural Erosion Symptoms Identification}
In the stage of detection, the latest researches give feedback about identified architectural erosion symptoms and their types during different project stages. However, the detected and named symptoms are oriented to Frontend and Backend development. For mobile development oriented to Android technology, specifically made in Kotlin programming language, it doesn't exist a repository of possible architectural erosion symptoms. The most recent researches use NLP techniques based on GitHub commits. The researchers analyze the main keywords written in GitHub commits that could indicate a bad architectural issue implementation, identifying a possible architectural erosion issue.
With a large amount of data from Git repositories is possible to make an NLP analysis of Github commits and define metrics that identify (from previously selected commits tagged by expert judges in software engineering and software architecture) similar keywords for architectural erosion. The metrics performance could be affected by the word embedding training context. The most recent research that implements that identification methodology, uses a Word Embedding model trained with 10 million of Stack overflow posts, a context that uses technical definitions for defining features, bugs, recommendations, and more of different programming languages. With this context, word embeddings could be more powerful and efficient for detecting architectural erosion issues keywords \cite{warnings-architectural-erosion,so-word-embedding}.
Another approach is similar to the one mentioned previously. However, those approaches are only based on architectural conformance checking. That consists of a set of different statements for a couple of judges who are professionals in software development and software architecture


\section{Metrics that could indicate Architectural Erosion}
Due to recent research, it is possible to determine the difference in metrics between an implemented architecture and an intended architecture, the main reason for the generation of the phenomenon of architectural erosion. In the resume, there are around 60 detected metrics in the software development process and in applications' source code analysis that could affect the maintainability of a software project. Research selected large software projects written in traditional programming languages like Python and Java and, with a set of judges, selected different source code implementations and past papers that try to describe and identify with similar methodologies. However, those metrics were detected in other development stacks like Frontend development and Backend development developed with traditional languages and frameworks. However, in the Android development context, the defined metrics could vary according to the mobile software development process for different reasons, like the recommended architectures for Android development and the mobile software development stages. For this reason, with the reviewed data for this research, we need to find different patterns that would indicate an architectural erosion issue and create a relationship between the reliability metrics found in recent research with the architectural erosion issues found in Android applications source code.
In this research, it was employed different papers that found a set of metrics in different analysis code platforms that use different methodologies, mainly static analysis code techniques. The research consists of a Systematic Literature Review (SLR) that found different studies from different research databases. The study found 43 relevant papers for architectural erosion metrics definition. Founded metrics are defined with different criteria statements like historical data revision, architectural complexity, architectural dependency coupling analysis, or architecture size analysis. These metrics were found from open-source software development projects to industrial software development projects.
Different measurement strategies established the effectiveness of the collected metrics. These strategies were found from different code analysis tools like SonarQube, CKJM, etc. Most of the detected issues are mapped with any nonfunctional requirement inside a software development project, but most of them are related to maintainability architectural issues and customized quality gates determined by architectural deterioration or evolution.

Furthermore, there are different metrics and reasons to handle good programming practices with different architectural standards. Based on the main quality attributes in software architecture, it is possible to determine control metrics that could identify improvements with the suppression of identified AER issues in the source code of an Android application.
In the availability quality attribute, one of the most common problems during the development stage of a software project is the bad implementation of exception handling. A bad exception handling in a software project could affect a complete flow that achieves a functional requirement and could generate catastrophic software failures and a possibility of a crash of an application. Despite this, some tools use custom lint check rules to detect common bad implementations of bad exception handling in software development. However, the tools spent considerable machine resources like computer processing time and RAM spent. \cite{handle-exceptions-references}

Additionally, the latency and scalability quality attributes have been implemented with asynchronous programming techniques for better performance in the Android operating system. The recommended architecture standards and guidelines give politics about the inappropriate use of blocking library functions for data or behavior change operations. The use of synchronous programming techniques in any layer based on the MVVM architecture could affect the use of resources of Android devices. For that, it is necessary to implement the use of coroutines techniques and avoid synchronous operation in mobile application development. In the detection of AER issues stage, it will be very important for testing AER rules and detecting blocking functions in Android source implementations \cite{performance-coroutines-reference}.


\section{Giving Architectural Erosion Solutions}
The latest researches give feedback about identified architectural erosion symptoms and their types during different realization project stages. However, the detected and named symptoms are oriented to Frontend and Backend development. For mobile development oriented to Android technology, specifically made in Kotlin programming language, it doesn't exist a repository of possible architectural erosion symptoms. To get this, we use an artificial intelligence approach for detecting architectural changes, and those changes and their change messages (commits in git world) are useful for identifying possible symptoms and generating rules to prevent them \cite{aer-metrics-paper}.


\subsection{Static Analysis Code techniques}
Architectural erosion is a general phenomenon that occurs in all the fields in the software development process. Recent research for architectural erosion identification and detection has been focused on Backend development and Frontend development. For this, different static analysis tools have been created as plugins of different  (Integrated Development Environment) IDEs. One example of that is the use of Antlr-determined grammars in the Eclipse plugin for bad pattern detection in (Data Transaction Object) DTO files. Those files on Backed development are used for service exposition and communication with other components inside a software project.
There are similar components that use static analysis code techniques used in the industry. The most common platforms are SonarQube and different linters with custom check lint rules.

With static analysis code techniques, different advantages exist for architectural erosion issue detection. One of them is the easiness of detecting patterns in terms of dependency classes, coupling components, class name standards, and package name standards. Furthermore, we can extend that detection approach to detect customized architectural rules for a specific software development process.\cite{master-thesis-aer-backend}

However, in the Android software development context, no platform considers the architectural erosion detection and identification process with architectural erosion metrics or customized quality gates. For this reason, it is necessary to define an identification process to detect different parents in different components. The pattern detection process must be based on a specific standard or specific recommended architectural pattern for Android application development. With the main concepts of programming languages, it is possible to define custom lint check rules with different tools like linters and IDE plugins. Static analysis code technique could help mobile developers to find architectural violations to defined nonfunctional requirements in the development stage.


\subsection{NLP techniques and AI Models}
For the architectural erosion identification process (and detection, but mainly AER identification process), different tools could be useful for architectural violation detection through Natural Language processing fundamentals. As an additional feature to that field, is possible to use different AI models powered by different training and contextualization techniques for getting a better performance in the AER issues detection process. Furthermore, the models present different ways to generate corrected code according to a detected issue in the AI training stage.

For the AER detection process, different AI models set have been implemented for the AER detection process based on developers' messages extracted from a code versioning platform like GitHub, GitLab, or OpenStack After a large identification process based on human judgments. With this identified AER issues dataset, basic AI models have been proved for potential key detection that could indicate architectural violations. 





