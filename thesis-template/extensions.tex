\chapter{Methodology Extensions}
\label{cha:extensions}

The explored solution approaches for AER issues identification could be extended to identify more issues related to different quality attributes in Android projects. As the developments mentioned in the related work, it is possible to use these methodologies to identify more issues in different layers of the application. The building of plugins and customized linter rules can be extended to identify different patterns to detect different issues in terms of security, availability, scalability, etc. In the same way, it is possible to use different AI models and NLP techniques to solve and detect issues according to the commit message. Nowadays, it is possible to generate a commit message according to the well-written commit standards. With those tools, we can explore the solution methodology approach to detect more issues and potential keywords related to a specific problem. We explored potential research fields whose proposed solution methodologies for AER issues identification could be extended and implemented.


\section{Static Code Analysis Methodology Extensions}
In the case of the use of static code analysis techniques, it is possible to create customized lint rules not only for a general standard of architecture or a specific architectural design pattern. Companies and tools (not only IDEs) can develop customized lint check rules for specific standards and patterns that the company of any specific situation.
We can explore the identification of different issues in different stages according to the components of the programming languages.

\section{Use of AI and NLP for Android Applications Issues Identification}
AI and NLP have extended the solution approaches in software engineering on a large scale. Different issues related to different quality attributes, policies, and standards of software architecture can be identified and possibly solved by the use of AI models. Furthermore, it is possible to use NLP techniques to identify potential words with a high semantical meaning. We identified possible extensions of the implemented methodology for identifying AER issues using AI models and NLP techniques.

\subsection{Design problems identified by NLP techniques and AI models}
 One possible extension is the identification of issues in the UI layer of an Android application. Nowadays, there are some standards and guidelines to develop a sustainable User Interface (UI). In terms of application design, there are two important things to develop a well-usable application: Accessibility and Internationalization. For some applications developed for Arabic countries, there are some issues related to the UI. For the Arabic language, some layouts do not display the text and information correctly. For social media applications, like TikTok, there are some issues related to accessibility. Some problems for blind people have been detected, the same for influencers with any disability. 
 This problem has been explored with an adapted methodology used for AER issues identification. It is possible to use NLP techniques to identify the most used words (after considering them as keywords for usability issues) in scientific articles and papers that made an analysis about usability issues in different Android applications. A word cloud was created wth the help of NLP techniques. After that, with current AI models and their integrated word embedding model, a clustering model was implemented to create groups of words that indicate a possible kind of usability issue. Different groups were detected for different kinds of usability issues, such as internationalization and audio issues. We can define a similar workflow to the one used for AER issues identification with AI models and NLP techniques. Furthermore, we can use the cosine similarity metric to identify potential keywords of Android applications in versioning platforms like GitHub.
 With a sample of around 35 applications and 65 papers, some words with an important semantical meaning for potential usability issues identification were detected. After that, a set of clusters was created to create categories for the different usability issues that could appear. Initially, it is possible to create a dataset of GitHub commits to explore some usability issues according to the identified potential keywords, to identify them with NLP techniques inside the commit message.