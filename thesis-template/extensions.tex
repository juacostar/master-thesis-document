\chapter{Methodology Extensions}
\label{cha:extensions}

The explored solution approaches for AER issue identification can be extended to uncover additional problems related to different quality attributes in Android projects. As demonstrated in the related work, these methodologies can be adapted to detect issues across multiple application layers. The development of plugins and customized linter rules can be expanded to identify patterns associated with concerns such as security, availability, scalability, and others. Similarly, various AI models and NLP techniques can be employed to detect and categorize issues based on commit messages. Modern tools even allow automatic generation of commit messages that follow well-established writing standards. By leveraging these capabilities, it becomes possible to further investigate and refine methodologies aimed at identifying additional issues and extracting meaningful keywords related to specific problem types.
In this research, we explored potential directions in which the proposed AER identification methodologies can be extended and implemented, demonstrating opportunities for broader detection of software quality concerns in Android applications.



\section{Static Code Analysis Methodology Extensions}
In the case of static code analysis techniques, it is possible to create customized lint rules not only for general architectural standards or specific architectural design patterns. Companies and development tools (not limited to IDEs) can define their own linting rules tailored to organizational standards and patterns required in specific development contexts. Additionally, the identification of issues can be explored across different stages of development by analyzing components of the programming language. Through the use of semantic pattern detection based on structures such as the Abstract Syntax Tree (AST) and the Call Graph, it is currently possible to evaluate numerous metrics that may signal issues affecting particular quality attributes in software projects. Based on this, for the identification of issues in Android applications, a set of detectable patterns can be defined in terms of syntactic and semantic characteristics of the programming languages used for Android development. From these patterns, customized lint check rules can be created and integrated into the development process. In this way, warnings and recommendations can be generated to support the resolution of multiple issues, such as those related to security, availability, connectivity, and other concerns discussed in the related work.


\section{Use of AI and NLP for Android Applications Issues Identification}
AI and NLP have extended the solution approaches in software engineering on a large scale. Different issues related to different quality attributes, policies, and standards of software architecture can be identified and possibly solved by the use of AI models. Furthermore, it is possible to use NLP techniques to identify potential words with a high semantical meaning. We identified possible extensions of the implemented methodology for identifying AER issues using AI models and NLP techniques.

\subsection{Design problems identified by NLP techniques and AI models}
One potential extension involves detecting issues within the UI layer of an Android application. Today, various standards and guidelines support the development of sustainable User Interfaces (UI). In terms of application design, two essential aspects contribute to a usable application: accessibility and internationalization. For instance, applications targeting Arabic-speaking regions sometimes experience UI problems, as certain layouts fail to correctly display information in right-to-left text. Similarly, social media platforms such as TikTok have demonstrated accessibility shortcomings, with challenges reported by blind users and content creators with disabilities. This problem has been investigated using an adapted version of the methodology applied to AER issue detection. NLP techniques can be used to extract and analyze frequently appearing terms—considered as potential usability-related keywords—from scientific literature discussing usability issues in Android applications. A word cloud was generated using standard NLP processing. With the help of modern AI models and their integrated word embedding representations, a clustering technique was applied to group semantically related terms, revealing categories of usability problems such as internationalization and audio-related issues. A workflow similar to the AER identification methodology can be designed, incorporating NLP techniques and AI-based similarity measures such as cosine similarity to detect additional relevant terms in versioning platforms like GitHub. Using a dataset consisting of approximately 35 applications and 65 research papers, several terms with strong semantic value for usability issue detection were identified. Clusters were then formed to categorize different types of usability concerns. Based on this, a dataset of GitHub commits can be created to investigate traces of usability problems in commit messages, allowing their detection through NLP techniques.
