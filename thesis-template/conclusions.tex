\chapter{Conclusions and Future Work}
\label{cha:conclusions}
We have explored mainly the problem of AER issues, focused on Android applications. We identified the problem and the suggested solutions in other software engineering areas and defined some research questions to explore this phenomenon in the Android environment. After that, we implemented and suggested some solution approaches, identified and adapted them into the Android context. For every solution approach, we analyzed and compared some results according to the experiments mentioned in the related work. Finally, we defined alternatives to extend the implemented methodologies for AER issues identification. Those alternatives are related to other quality attributes in software architecture and issues found in specific layers of any architecture (like the UI layer of any Android project). Finally, with this work, we can define a conclusion related to the main scope of the research and the defined research question.

\section{Research Questions conclusions}

\begin{itemize}
	\item  \textbf{How can we identify Architectural Erosion in Android apps? Architectural erosion can be identified in different ways. The most well-known identification methodologies of AER are powered by static code analysis tools and the use of AI models and NLP techniques. With these methodologies, we can identify some issues and architectural violations related to one or more architectural guidelines and standards

\end{itemize}

