\chapter{Research Questions and Scope}
\label{cha:objectives}

According to the research problem about AER issues in Android projects, it is possible to define a set of desired results and objectives about how to explore issues related to AER based on the implemented methodologies in other kinds of research. The main idea is to establish the goals of every method. For the definition of this, we define a set of research questions related to the advantages of identifying AER issues in Android projects and the extensions for solving issues related to other quality attributes, specifically in Android projects.

\section{Research Questions}

\begin{enumerate}
	\item \textbf{How can we identify Architectural Erosion in Android apps?} The desired result of answering this question is to identify the standards, strategies, and policies for finding AER issues in software engineering. According to the research, we can identify the most important approaches to identifying AER issues.
	\item \textbf{What methodologies can we use to detect AER in Android apps?} The main idea is to explore the most well-known methodologies about AER issues in software engineering in general. We will adapt and extract the main features of each methodology to set them to find those issues in Android projects.
	\item \textbf{How effective are the proposed methodologies?} With the implemented methodologies for finding AER issues in Android projects, we will test with different approaches to measure the effectiveness of each one. These results will conclude what the best methodology is to identify AER issues in Android projects. Furthermore, it would define future work focused on detecting more issues that could affect an Android application in general. The proposed methodologies would be helpful to get better applications and better development standards, supporting software sustainability.
\end{enumerate}

\section{Research Scope}
With the defined research questions, we can define the desired results and goals of this research. These must be defined within the scope of the research: \textbf{The definition of methodologies to identify Architectural Erosion in Android apps}. Based on this, we define additional objectives to support the main scope of the research.

\begin{enumerate}
	\item Define methodologies based on the well-known methodologies applied in software applications. With these, we can implement for everyone in any Android project to detect and explore issues in the development stage.
	\item Propose standards and policies about identifying AER issues in Android apps. Standards could improve the development of plugins and tools for AER issues detection using different approaches.
	\item Test and analyze the results of the implementations of the different studied methodologies.
\end{enumerate}


The research questions and objectives can define a strategy to implement the main scope of the research and identify improvements in finding AER issues in Android apps and their associated strategies.