% $Id: introduction.tex 1784 2012-04-27 23:29:31Z nicolas.cardozo $
% !TEX root = main.tex

\chapter{Introduction}
\label{cha:introduction}



At the beginning of system development, teams often conduct a thorough design process, where they define the system components and resources needed for the app. This process is named \emph{software architecture definition}. After that, teams have to align the development process with those architectural rules to achieve their goals in performance, security, availability, and other quality attributes. However, due to constant pressure for releases and frequent feature requests, developers tend to deviate from the \textit{software architecture definition}, affecting system quality attributes. This could generate issues in security, availability, performance, and latency. This deviation is called \emph{architectural erosion}. In mobile development, it can reduce application performance and other system quality attributes, affecting device resources and, ultimately, the user experience.

Previous efforts have begin focused on functional and non-functional requirements, such as security, eventual connectivity, and code smells. However, to the best of our knowledge, there is no previous work focused on analyzing the deviation of the development process of android apps from its defined architecture. This study plans to evaluate the efficiency of two proposed methodologies to identify and locate architectural erosion bugs. 

%%
\section{Motivation}
Previous research has shown that the costs associated with human resources, technology, time, and money tend to increase after the first release and deployment of a software project\misref. Therefore, it becomes crucial to identify the factors and artifacts that influence software sustainability and performance in the short, medium, and long term. To address this, various approaches analyze different components of software projects, including their design, source code, and architectural rules.
Among these, static code analysis has been one of the most widely adopted and extensively studied\misref. This approach relies on tools that detect and analyze instances of architectural erosion in specific code fragments, often tracking their evolution over time. While these tools have proven highly valuable for developers and teams, their application in mobile development remains limited. Mobile ecosystems, such as Android, operate under stricter resource constraints compared to server-side environments, which makes early detection and recovery from architectural erosion even more critical.
Therefore, there is a need to design specialized components that can identify architectural erosion within Android applications and assist developers in addressing these issues proactively. Another promising line of research involves using Natural Language Processing (NLP) techniques to analyze commit messages from version control platforms across Android projects, enabling the detection of patterns related to architectural erosion.

%%%%
\section{Machine learning models in architectural erosion}
Recent research in Machine Learning (ML), particularly in Natural Language Processing (NLP), has shown that pre-trained models used for word classification or generation tend to perform better with programming languages than with natural languages such as English or Spanish. This advantage arises from the structured and well-defined syntax of programming languages. Such insights can be leveraged in software repository mining to detect issues in source code. Indeed, problems such as masked code and design pattern detection have already been addressed using pre-trained ML models.
In the context of architectural erosion, these models could play a crucial role in identifying early indicators of degradation. Once such symptoms are detected, more specific static code analysis rules can be applied to confirm and address the architectural deviations. In summary, ML models have the potential to serve as an early warning system for architectural erosion in Android projects, leading to the development of optimized components for early detection and recovery.

The objective of this research is to explore two methodologies for identifying Architectural Erosion-Related (AER) issues in Android projects. The first is based on Static Code Analysis techniques, aiming to identify existing works and tools that apply such methods to detect erosion. The second leverages Artificial Intelligence (AI) and NLP models to analyze large collections of commits extracted from GitHub repositories of Android projects. The results obtained from both methodologies will be compared and evaluated to assess their effectiveness and complementarity in detecting AER issues.

Furthermore, this research will examine how these methodologies can be extended to address additional quality attributes in Android applications, such as usability. We propose extending the NLP- and AI-based approach to detect usability-related issues in Android apps, alongside using static code analysis tools tailored for this purpose. Finally, we will explore data-driven analysis of app reviews to identify recurrent usability problems and other quality issues emerging over time.

\endinput

