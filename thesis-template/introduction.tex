% $Id: introduction.tex 1784 2012-04-27 23:29:31Z nicolas.cardozo $
% !TEX root = main.tex

\chapter{Introduction}
\label{cha:introduction}

At the beginning of system building, we might undergo a strong design process, where we define the system components and resources that we need to use. This process is named \emph{software architecture definition}. After that, you have to align the development process with those architectural rules to achieve goals in performance, security, availability, and others.
However, due to growth in software development, as time goes by, software tends to violate architectural rules, affecting system quality attributes. This could generate issues in security, availability, performance, and latency. This deviation is called \emph{architectural erosion}. In mobile development, it can reduce application performance and other system quality attributes, affecting device resources and, ultimately, the user experience.
In recent research, there have been significant advances in bug resolution across different approaches. For example, in security, connectivity, and code smells, there are various tools and components that detect and offer solutions for different bugs related to these approaches.

However, in the architectural erosion analysis, despite some research tackling this concept, some of these propose a toolset; it is not clear how we can resolve architectural erosion insights or a set of directives for facing this problem in the mobile development ecosystem. For this reason, it is important to find the impact of architectural erosion in mobile applications and how we can identify and locate architectural erosion bugs with two proposed methodologies:  

%%
\section{Why make research efforts in Architectural Erosion Identification?}
Due to different research, the costs in terms of human resources, technological resources, time, and money would grow after the first release and deployment of the software project. By this, it is necessary to establish the different reasons and artifacts that affect the software sustainability, the software performance in the short, middle, and long term. For this task, many approaches analyze different software project components like its design, its code, and its architectural rules. One of the most important, and one of the most studied, too. It has been the static code analysis approach. In this approach, a lot of tools have been created to identify and detect architectural erosion in specific code fragments, showing an analysis over time. These tools have been very helpful for the developers and for the teams, but it has not been enough in the mobile development ecosystem, where system resources are more limited than in the side-server environment. It is necessary to build a component with a specific purpose in Android Apps and show the importance of detecting architectural erosion in mobile app code and recovering these issues in an early way. Another approach is the identification with Natural Language Processing (NLP) techniques in commits of versioning platforms in a set of Android projects.

%%%%
\section{Machine learning models in architectural erosion}
Recent research in Machine Learning models, specifically in Natural Language Processing, has concluded that the use of pre-trained models for the classification or generation of words in a specific language, their performance with programming languages is better than natural languages (in this NLP approach, English, Spanish, etc), according to their syntax. This idea could be recreated in issue detection in source code repository analysis. Issues like masked code, pattern design detection, and others have been implemented with pre-trained machine learning models. In architectural erosion, it could be useful for detecting the first symptoms of this and making early detection for, after that, using specific rules of architecture in static code analysis. In summary,  machine learning models could detect the first symptoms of architectural erosion in an Android project and create a component with optimized performance for detecting and recovering issues about this.

The objective of this research is to explore two identification methodologies of AER issues applied in Android projects. The first one is based on Static Code Analysis techniques. The main objective is to identify implemented work using the proposed techniques according to this methodology. The second one is based on the use of Artificial Intelligence (AI) models and NLP fundamentals. The objective of using this methodology is to analyze a large set of extracted commits from GitHub of a set of Android projects. The results of both methodologies are tested with judgments and comparisons of the results, and the use of AI techniques.

Furthermore, we will explore the alternatives of the use of these methodologies for solving more issues related to specific quality attributes in Android projects, like usability. We will extend the methodology based on NLP and AI models to explore the detection of usability issues in Android apps and the use of different tools based on Static code analysis to detect usability issues in Android projects.  Finally, we explore the data analysis of reviews of Android apps as an extension to identify prominent issues based on the Applications and their reviews over time.

\endinput

