% $Id: introduction.tex 1784 2012-04-27 23:29:31Z nicolas.cardozo $
% !TEX root = main.tex

\chapter{Introduction}
\label{cha:introduction}



At the beginning of system development, teams often carry out a thorough design process in which they define the system components and the resources required for the application. This process is known as *software architecture definition*. Afterward, teams need to align the development process with the established architectural rules in order to meet their goals in performance, security, availability, and other quality attributes. However, due to constant pressure to deliver new releases and frequent feature requests, developers often deviate from the software architecture definition, negatively impacting these system quality attributes. Such deviations can lead to issues related to security, availability, performance, and latency. This phenomenon is referred to as *architectural erosion*. In mobile development, architectural erosion can reduce application performance and degrade other system quality attributes, affecting device resources and ultimately harming the user experience.

Previous efforts have focused on functional and non-functional requirements, such as security, intermittent connectivity, and code smells. However, to the best of our knowledge, no prior work has focused on analyzing the deviation of the development process of Android applications from their defined architecture. This study aims to evaluate the effectiveness of two proposed methodologies for identifying and locating architectural erosion issues.


%%
\section{Motivation}
Previous research has shown that the costs associated with human resources, technology, time, and budget tend to increase after the first release and deployment of a software project \citet{bass-architecture-book}. Therefore, it is crucial to identify the factors and artifacts that influence software sustainability and performance in the short, medium, and long term. To this end, various approaches analyze different components of software projects, including their design, source code, and architectural rules.
Among these, static code analysis has been one of the most widely adopted and extensively studied \citet{ieee-automated-detection}. This approach relies on tools that detect and analyze instances of architectural erosion in specific code fragments, often tracking their evolution over time. Although these tools have demonstrated significant value for developers and teams, their application in mobile development remains limited. Mobile ecosystems, such as Android, operate under stricter resource constraints compared to server-side environments, which makes early detection and recovery from architectural erosion even more critical.
As a result, there is a need to design specialized mechanisms capable of identifying architectural erosion within Android applications and assisting developers in addressing these issues proactively. Another promising research direction involves applying Natural Language Processing (NLP) techniques to analyze commit messages from version control platforms across Android projects, enabling the detection of patterns related to architectural erosion.


%%%%
\section{Machine learning models in architectural erosion}
Recent research in Machine Learning (ML), particularly in Natural Language Processing (NLP), has shown that pre-trained models used for word classification or generation tend to perform better with programming languages than with natural languages such as English or Spanish. This advantage stems from the structured and well-defined syntax of programming languages. These insights can be applied to software repository mining to detect issues in source code. In fact, problems such as masked code and design pattern detection have already been addressed using pre-trained ML models.

In the context of architectural erosion, such models could play a crucial role in identifying early indicators of degradation. Once these symptoms are detected, more specific static code analysis rules can be applied to confirm and address the architectural deviations. In summary, ML models have the potential to serve as an early warning system for architectural erosion in Android projects, leading to optimized components for early detection and recovery.

The objective of this research is to explore two methodologies for identifying Architectural Erosion-Related (AER) issues in Android projects. The first is based on Static Code Analysis techniques, aiming to identify existing work and tools that apply such methods to detect erosion. The second leverages Artificial Intelligence (AI) and NLP models to analyze large collections of commits extracted from GitHub repositories of Android projects. The results of both methodologies will be compared and evaluated to assess their effectiveness and complementarity in detecting AER issues.

Furthermore, this research will examine how these methodologies can be extended to address additional quality attributes in Android applications, such as usability. We propose extending the NLP- and AI-based approach to detect usability-related issues in Android applications, alongside using static code analysis tools tailored for this purpose. Finally, we will explore data-driven analysis of app reviews to identify recurrent usability problems and other emerging quality issues over time.
\endinput

